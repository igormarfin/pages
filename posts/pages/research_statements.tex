\documentclass[a4paper]{article}
% generated by Docutils <http://docutils.sourceforge.net/>
\usepackage{fixltx2e} % LaTeX patches, \textsubscript
\usepackage{cmap} % fix search and cut-and-paste in Acrobat
\usepackage{ifthen}
\usepackage[T1]{fontenc}
\usepackage[utf8]{inputenc}
\setcounter{secnumdepth}{0}
\usepackage{tabularx}

%%% Custom LaTeX preamble
% PDF Standard Fonts
\usepackage{mathptmx} % Times
\usepackage[scaled=.90]{helvet}
\usepackage{courier}

%%% User specified packages and stylesheets

%%% Fallback definitions for Docutils-specific commands

% providelength (provide a length variable and set default, if it is new)
\providecommand*{\DUprovidelength}[2]{
  \ifthenelse{\isundefined{#1}}{\newlength{#1}\setlength{#1}{#2}}{}
}

% docinfo (width of docinfo table)
\DUprovidelength{\DUdocinfowidth}{0.9\textwidth}
% numeric or symbol footnotes with hyperlinks
\providecommand*{\DUfootnotemark}[3]{%
  \raisebox{1em}{\hypertarget{#1}{}}%
  \hyperlink{#2}{\textsuperscript{#3}}%
}
\providecommand{\DUfootnotetext}[4]{%
  \begingroup%
  \renewcommand{\thefootnote}{%
    \protect\raisebox{1em}{\protect\hypertarget{#1}{}}%
    \protect\hyperlink{#2}{#3}}%
  \footnotetext{#4}%
  \endgroup%
}

% inline markup (custom roles)
% \DUrole{#1}{#2} tries \DUrole#1{#2}
\providecommand*{\DUrole}[2]{%
  \ifcsname DUrole#1\endcsname%
    \csname DUrole#1\endcsname{#2}%
  \else% backwards compatibility: try \docutilsrole#1{#2}
    \ifcsname docutilsrole#1\endcsname%
      \csname docutilsrole#1\endcsname{#2}%
    \else%
      #2%
    \fi%
  \fi%
}

% hyperlinks:
\ifthenelse{\isundefined{\hypersetup}}{
  \usepackage[colorlinks=true,linkcolor=blue,urlcolor=blue]{hyperref}
  \urlstyle{same} % normal text font (alternatives: tt, rm, sf)
}{}
\hypersetup{
  pdftitle={Research statements},
  pdfauthor={Igor Marfin}
}

%%% Title Data
\title{\phantomsection%
  Research statements%
  \label{research-statements}}
\author{}
\date{}

%%% Body
\begin{document}
\maketitle

% Docinfo
\begin{center}
\begin{tabularx}{\DUdocinfowidth}{lX}
\textbf{Date}: &
	2013-09-04 23:37:47 \\
\textbf{category}: &
HOW-TO
\\
\textbf{tags}: &
how-to
\\
\textbf{Author}: &
	Igor Marfin \\
\end{tabularx}
\end{center}

During my  Ph.D. work  at the Deutsches Electronen-Synchrotron with the CMS
experiment at the Large Hadron Collider (LHC),
I have witnessed one of the world's most
complex scientific undertakings during its final construction, commissioning, operation, and
announce of its first discovery. It has been an exciting time.

The Compact Muon Solenoid (CMS) \raisebox{1em}{\hypertarget{id1}{}}\hyperlink{ref1}{[1]} is one of the two general purpose
detectors at the LHC. CMS was  built to detect the products of the
proton-proton collisions with high energies of tens TeV.
The LHC had a tremendous success last years. Besides many important
measurements of the standard model (SM) physics and a wide spectrum of searches for new
physics beyond the SM, a very important and exciting discovery of a new boson with mass
of 125 GeV has been recently made \raisebox{1em}{\hypertarget{id2}{}}\hyperlink{ref2}{[2]}.
Measuring the properties of the Higgs boson could be the first step
verifing the Higgs mechanism to break the electroweak symmetry of SM.
An important study that would
allow to understand the properties of the newly discovered boson is a
measurement of the coupling strength between the boson and third-generation SM particles:
top,bottom quarks and tau lepton.
A precise study of decays
\DUrole{raw-tex}{$H\rightarrow \tau\tau $} and \DUrole{raw-tex}{$H\rightarrow b\bar{b} $}
would play a crucial test of the Higgs mechanism in the SM or
in the theories beyond it.

Even after a great experimental success, the SM suffers from some
theoretical problems such as: instability
of the Higgs boson mass against radiative correction, hierarchy of the fundamental scales, hierarchy
in the masses of fermions, mechanism to break the electroweak symmetry, mechanism of CP violation
etc. Some or all of these problems are addressed in various models of physics beyond the SM, such as
models with extra-dimensions, with supersymmetry, or with other additional symmetries. Each of these
model invariably propose new heavy particles that can potentially be produced at the LHC.

I am interested in pursing physics involving the Higgs coupling with the third-generation
SM particles like top,bottom quarks and tau lepton. I am confidient that my
extensive research experience allows me significantly contribute in these efforts.
I already play an important role in studying the Higgs physics  with b jets, which I perform within
my doctoral scholarships in DESY CMS Higgs group.  This activity is complement by
my expertise of the online procedure, the High Level Trigger \raisebox{1em}{\hypertarget{id3}{}}\hyperlink{ref3}{[3]},
filtering multi-jet events
in the CMS detector. I am one of the primary authors of the search for the neutral Higgs boson decaying
to pair of \DUrole{raw-tex}{$b\bar{b}$}  quarks in association with b-quark  performed at 7 TeV
data. The analysis covers the neutral Higgs sector in minimal SUSY scenarios \raisebox{1em}{\hypertarget{id4}{}}\hyperlink{ref4}{[4]}.
One of my primary contributions was a Monte Carlo study of the signal and background kinematics
to optimize the offline pre-selection  criteria used in the analysis. This yields
an sensitivity improvement of the search potential in this challenging channel.
I have developed the method of multivariate analysis to optimize the Higgs search procedure
in a unbiased way.
I have further expanded my research experience investigating the estimations of
effects from the  systematic uncertainties  related to the analysis.
I have played the main role in
obtaining the final results, developing
a fitting procedure used to extract the  observed cross section of the Higgs boson production.
Using statistical procedures,
I have performed the calculation of the exclusion limits on the parameters in the Supersymmetric theory.

The large hadronic interaction rate at the LHC  poses a great challenge
for triggering the events with
signature  of three b jets in the final state.
A significant reduction of the rates with preserving the signal efficiency
is achieved  with b-jet identification algorithms (b-tagging) applied at the HLT.

To ensure high efficient performance of b-tagging at the HLT,  I
serve as one of the active developers of such algorithms in the CMS Trigger Study Group.
In particular, one of my primary responsibility is to superviese
and control such efforts in the online b-tagging optimization.
All  improvements of b-jet tagging at the HLT are important not only for the
Hbb production process but also for  boosted Higgs, and a number of beyond
SM searches that have b jets in the final state.
The current upgrade of the b-tagging  HLT will guaranty that the CMS detector performance
and data analysis methods are optimal for the high-intensity, high pileup
environment of the next LHC run at the centre-of-mass energy of 14 TeV.

Due to my experience , I have in the \DUrole{raw-tex}{$bH\rightarrow 3b$}  search
and in b-jet identification, there is an obvious opportunity for me
to bring my ideas to the PUT\_NAME\_OF\_UNI \raisebox{1em}{\hypertarget{id5}{}}\hyperlink{ref5}{[5]} and its efforts with the CMS
experiment.

The first task in any collider data analysis, and especially in the search for new physics, is to ensure
that the relevant collisions are recorded for later analyses. During the first
half of the fellowship, I propose to continue to develop my current work within the
CMS  trigger. My position as trigger software validation co-coordinator will allow me to use the lessons
learned from the  collected data  in 2012 (2011) to identify
the directions for the software upgrade during the current long shutdown.
The strong trigger involvement of the  PUT\_NAME\_OF\_UNI \raisebox{1em}{\hypertarget{id6}{}}\hyperlink{ref5}{[5]}
would make it an ideal place to develop
this activity.  I plan to clarify  the strategic trigger needs of the Higgs analyses  in 2014.
In addition to that, the set of the control triggers will be added to estimate efficiencies of the
Higgs triggers   in the unbiased way.

The second half of the fellowship will correspond to a period when the LHC will run with high
luminosity at 14 TeV energy. During this period,
I propose  to contribute   to measurements of production cross section   for \DUrole{raw-tex}{$b\bar{b}H$} and \DUrole{raw-tex}{$t\bar{t}H$}.
This analysis will rely on  the triggers with btagging for which  the efficiencies are estimated from
the previous step of the fellowship. My experience of the trigger software will be hugely valuable me in this task.

After obtaining the first measurement of the production cross section,
the \DUrole{raw-tex}{$H\rightarrow b\bar{b}$} and \DUrole{raw-tex}{$H\rightarrow \tau\bar{\tau}$}
would be ideal  observable channels of the Higgs
boson coupling to the third-generation fermions.  The next step will be to measure
such couplings of the Higgs boson candidate,
and so to verify whether it corresponds to the SM prediction, or if it is part of a more complex scenario.
The LHC Higgs Cross Section Working Group has
recently published recommendations for the parametrization of measurements to explore the
couplings of this new Higgs-like state  \raisebox{1em}{\hypertarget{id7}{}}\hyperlink{ref6}{[6]}. For example,
it can be achived by observing the Higgs in the channels, and
the ratios of Higgs production cross sections in these channels.

If  the evidence for a Higgs boson in the
\DUrole{raw-tex}{$$t\bar{t}H,\,\, b\bar{b}H,\,\, H\rightarrow \tau\bar{\tau},b\bar{b}$$}  would be found,
I intend to measure its CP quantum number \raisebox{1em}{\hypertarget{id8}{}}\hyperlink{ref7}{[7]}.
The observation of a CP-odd Higgs candidate, would exclude the SM Higgs, which
is CP-even.
If the studying the properties of the Higgs sector would demonstrate that the current
Higgs excess is not SM-like, it would offer the first convincing evidence of supersymmetry.
The observables of CP quantum number  involve the polarization and/or spin correlation of the decay products
of the Higgs boson and the associated heavy flavour, top or bottom, quark(s).
The polarization observables are strongly related to the azimuthal distribution of
secondary lepton from decay of the top quark  or the Higgs boson.
The polarization of the produced particles are usually studied by means of kinematical   distributions.

While the outlined research activity to measure the Higgs properties
would likely take a few years, I also would like to get engaged in
the CMS upgrade activity to extend my research experience.
It is very important  to have a
hands-on experience with the hardware which helps me to efficiently participate in
the  CMS detector operation during  the upcoming LHC runs in 2014.

In summary, I would be enthusiastic to contribute to any efforts to measure
all of the couplings of the new resonance. My strong experience of
the CMS HLT  and  data analysis tools, my knowledge and publications in wide variety of topics
covering the theoretical and experimental physics aspects
makes me well-prepared to tackle the very exciting investigation of
the electroweak symmetry breaking and its origin in the LHC data.
Working in such a
prestigious laboratory with a team of brilliant and motivated people is a huge opportunity.
A post-doctoral position at PUT\_NAME\_OF\_UNI \raisebox{1em}{\hypertarget{id9}{}}\hyperlink{ref5}{[5]}  is an excellent match for my research,
As a post-doctoral researcher, I will endeavor to distribute an atmosphere where
knowledge and experience are easily shared among collaborators, and to continue to take
initiative in leading projects.
%
\DUfootnotetext{ref1}{id1}{{[}1{]}}{\phantomsection\label{ref1}%
ref1
}
%
\DUfootnotetext{ref2}{id2}{{[}2{]}}{\phantomsection\label{ref2}%
ref2
}
%
\DUfootnotetext{ref3}{id3}{{[}3{]}}{\phantomsection\label{ref3}%
ref3
}
%
\DUfootnotetext{ref4}{id4}{{[}4{]}}{\phantomsection\label{ref4}%
ref4
}
%
\DUfootnotetext{ref5}{id5}{{[}5{]}}{\phantomsection\label{ref5}%
ref5
}
%
\DUfootnotetext{ref6}{id7}{{[}6{]}}{\phantomsection\label{ref6}%
ref6
}
%
\DUfootnotetext{ref7}{id8}{{[}7{]}}{\phantomsection\label{ref7}%
ref7
}

\DUrole{raw-tex}{\newpage}


\section{Appendix%
  \label{appendix}%
}


\subsection{How to generate the document%
  \label{how-to-generate-the-document}%
}
%
\begin{quote}{\ttfamily \raggedright \noindent
rst2latex.py~research\_statements.rst~\DUrole{se}{\textbackslash{}\\
}~-{}-footnote-references\DUrole{o}{=}brackets~>~research\_statements.tex\\
pdflatex~research\_statements.tex\\
evince~research\_statements.pdf
}
\end{quote}

\end{document}

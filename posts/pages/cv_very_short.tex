

\vspace{4mm}

{\normalsize{

\begin{tabular}{lll}
Name: &  & Igor B. Marfin \\
      &  & \\
Date of birth    & & 30 March 1979 \\
Place of birth   & & Minsk, Belarus \\
Marital status   & & married \\
Profession       & & physicist \\
Spoken languages & & English (fluent), German (sufficient), \\
                 & & Russian (native), Belorussian(native)  \\
\end{tabular}

\vspace{3mm}

\begin{center}{\bf{{\underline{Education}}}}\end{center} 

\begin{tabular}{ll}
1986-1996      & Minsk Gymnasium N3,   specialized in economics.   Graduated  in June 1996.      \\
1996-2001      & Belorussian State University,  Faculty of Physics, Minsk, Belarus,       \\
               & undergraduate studies.   Graduated with honors in June 2001.      \\
2001-2005      & National Center of Particle and High Energy Physics,\\
		       & Minsk, Belarus, group of Prof. N. Schumeiko,  post-graduate studies,                         \\
2005-2008	   & Work under INTAS Young Scientist Grant, \\
2008-2011	   & DESY-Zeuthen, group of Prof. W. Lohmann, \\
2011-2013      & DESY CMS Higgs group, YIG  of Dr. A. Raspereza\\


\underline{M.Sc. thesis} & ``Deep inelastic scattering leptons \\
		& at nucleons within the  Standard model \\ 
		& of electroweak interaction'' \\
               & Defended with distinction in June 2001\\

\underline{Ph.D. thesis} & ``Search for new physics \\
                & with multi b-quark final states at LHC'' \\
                & to be submitted by November of 2013 \\

\end{tabular}

\vspace{3mm}

\begin{center}{\bf{{\underline{Professional skills}}}}\end{center} 

\begin{itemize}
\item{Programming with C, C++, FORTRAN, JAVA, Perl, Python languages} 
\vspace{-2mm}
\item{Detector simulation with  Geant4 packages}
\vspace{-2mm}
\item{Object-oriented design of the reconstruction software for particles  physics experiments }
\vspace{-2mm}
\item{Object-oriented design of adaptive Monte Carlo generators}
\vspace{-2mm}
\item{Developing the modules and plugins for algebraic computation at MATHEMATICA program}
\vspace{-2mm}
\item{Calculation  of radiative effects in perturbative quantum field theories}
\vspace{-2mm}
\item{Statistical analysis with exclusion limits, profile likelihood fits etc }
\vspace{-2mm}
\item{Multivariate analysis}
\vspace{-2mm}
\item{Numerical analysis}
\vspace{-2mm}
\end{itemize}

\newpage

\begin{center}{\bf{{\underline{Work experience}}}}\end{center} 

\vspace{2mm}

\begin{tabular}{lll}
Period         & Institute / Position            & Activities                     \\
\hline
2001-2005      & National Institute     &  The precise study of  gauge boson      \\
               & for Particle and High  &  interactions in photon-photon          \\
               & Energy Physics         &  reactions in the framework             \\
               & Minsk, Belarus,        &  of SM and beyond on ee-linear and pp-  \\
               & Junior researcher      &  colliders by means of two-photon       \\
               &                        &  processes.                             \\
               &                        &  High accuracy analysis of              \\
               &                        &  radiative corrections to WW pair       \\
               &                        &  gauge boson production in              \\ 
               &                        &  gamma-gamma collisions.                \\
	       & 			&  Investigation of Anomalous Quartic     \\
	       &			&  Gauge  boson Coupling (AQGC)           \\
	       &			&  Membership in the TESLA		  \\
	       &			&  Working Group			  \\
	       &			&  for Electroweak Physics		  \\
\hline
2005-2008      & National Institute     &  Analysis of process like               \\ 
               & for Particle and 	&  $pp\rightarrow\gamma\gamma\rightarrow WWX1X2$   \\ 
	       & High Energy Physics,        &  with AQGC at CMS.                      \\
               & Minsk, Belarus,        &  Membership in the RDMS CMS 		  \\ 
               & CERN,                  &  Working Group for Electroweak Physics. \\
               & Geneva,Switzerland,    &  Work under INTAS Young Scientist Grant,\\
               & Researcher             &  no. INTAS-YS-05-112-5429		  \\ 
\hline
2008-2011      & DESY-Hamburg,		&  Calculation of $O(\alpha)$ electroweak    \\ 
               & ZEUS group    		&  radiative  correction of to            \\ 
               & Hamburg, Germany,	&  Charge Current Deep in elastic         \\
               & DESY-Zeuthen,     	&  scattering of $e^{\pm}$ on protons in  \\ 
               & CMS group      	&  ZEUS experiment at HERA.               \\
               & Zeuthen, Germany,      &  Developing $pp\rightarrow t\bar{t} + jets$ events selection.  \\
               & Researcher             &  Study of systematic uncertainty  on    \\
               &                        &  ttbar production.                      \\ 
               &                        &  Study of muons from real data           \\
               &                        &  (900 GeV and 2.3 TeV) at CMS.          \\
               &                        &  Developing methods of real data  	  \\
	       &			&  estimation of efficiencies  		  \\
	       &			&  for btagging algorithms.		  \\
\hline
2011-2013      & DESY CMS Higgs group	&  The model-independent analysis of    \\ 
               & Hamburg, Germany,      &    $H+b\rightarrow 3b$                 \\ 
               & DESY-Zeuthen,     	&  channel at 7+8 TeV data  \\ 
               & CMS/FCAL group      	&  recorded by CMS at LHC in 2011 and 2012. \\
               & Zeuthen, Germany,      &  The statistical interpretation of   \\
               & PhD student of the supervisors    &   $H+b\rightarrow 3b$ results in   \\
               & Prof. W. Lohmann and   &  the MSSM.                      \\ 
               & Dr. R. Walsh           &  Developing and tuning MVA methods         \\
               &                        &  for the blinding policy          \\
               &                        &  of the analysis  	  \\          	
\hline
2013           & convener of B-tag POG  &  Responsibility for  Btagging in \\ 
               &   in CMS, CERN         &  CMS  HighLevel Trigger          \\

\end{tabular}

\newpage

\begin{center}{\bf{{\underline{Teaching experience}}}}\end{center} 

\vspace{1mm}

\begin{tabular}{lll}
Period               & Institute                      &    Subject                         \\
\hline               
2001                 & Belorussian State University,  & Practical lessons on               \\
(winter/summer       & Minsk Belarus                  & particle and high energy physics   \\
 semesters)          &                                &                                    \\
                     &                                &                                    \\
2002                 & Belorussian State University,  & Practical lessons on               \\
(winter/summer       & Minsk Belarus                  & quantum mechanics                  \\
semester)	     &				      &  	                           \\
                     &                                &                                    \\
2011                 & DESY,                          & Advisor in   DESY                \\
(summer              &  Zeuthen Germany               & Summer Student Programme          \\
semester)	     &				      &  	                           \\
                     &                                &                                    \\
2012                 & DESY,                          & Advisor in   DESY                \\
(summer              &  Zeuthen Germany               & Summer Student Programme          \\
semester)	     &				      &  	                           \\
                     &                                &                                    \\
2013                 & DESY,                          & Advisor in   DESY                \\
(summer              &  Zeuthen Germany               & Summer Student Programme          \\
semester)	     &				      &  	                           \\
                     &                                &                                    \\
2013                 & Brandenburg Technical University,  & Several lectures on                \\
(winter              &  Cottbus Germany                   & quantum mechanics, field theory       \\
semester)	     &				      &  and cosmology 	         \\



\end{tabular}



\begin{center}{\bf{\underline{Awards and Honors}}}\end{center}
%
\begin{itemize}
\item{INTAS Young Scientist Grant, no. INTAS-YS-05-112-5429, CERN} 
\item{Awarded by "Wilhelm und Else Heraeus" foundation  in recognition for the contributions made to the  
DPG Spring Meeting 2013 in Dresden }
%\item{Existential philosophy, literature and poetry}
%\item{Ancient and medieval history}
\end{itemize}

\vspace{1mm}

\begin{center}{\bf{{\underline{Reference letters are available upon request from:}}}}\end{center} 

\begin{tabular}{l}
Dr. R. Walsh, \\
DESY, Notketsrasse 85, D-22607 Hamburg Germany; \\
\\
Prof. Dr. Wolfgang Lohmann \\
DESY-Zeuthen, Platanenalle 6, D-15738 Zeuthen, Germany \\
\end{tabular}

\vspace{2mm}


\begin{center}
\begin{tabular}{ll}
E-mail           & Igor.Marfin@desy.de	\\ 
		 & marfin@cern.ch \\
                 & \\
Current Address  & DESY-Zeuthen	 \\
                 & Platanenalle 6, \\
                 & D-15738 Zeuthen, Germany \\
\end{tabular}
\end{center}











